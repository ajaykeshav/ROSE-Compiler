\chapter{Using Compass GUI}

Compass has a GUI available for exploring the checker warnings for either the compilation
of a single source file or the compilation of a whole project. This GUI allow the user to
interactively select checkers, run those checkers on the source file(s) of 
interest and display the source location of each violation. As a user convenience the
interface will either display the violating source region in a text editor or a 
non-editable display window.

The Compass GUI is located in the 'projects/compass/tools/compass/gui' directory of the
ROSE distribution. The Compass GUI is build as part of the standard Compass build when
ROSE is configured with qt4. In order to enable simple compilation of whole projects using
one-button clicks in the Compass GUI ROSE must be configured with sqlite3 as well.

\section{Running Compass GUI on a Single File}
\label{runningCompassGUISingleFile}

Before running the Compass GUI on a single file  the files listed 
in \ref{usingCompassInstallation} must be available and the environment variable 
'\$COMPASS\_PARAMETERS' must specify which compass\_paramaters file to use. CompassGUI can
then be invoked as a normal compiler like e.g:
\begin{verbatim}
compassMainGui -o ex1 ex1.C
\end{verbatim}
If it is not desirable to export the '\$COMPASS\_PARAMETERS' variable a shorthand is:
\begin{verbatim}
env COMPASS_PARAMETERS=/location/of/compass_pramateres   compassMainGui -o ex1 ex1.C
\end{verbatim}

\section{Running Compass GUI on an Autotools Project}

A goal of this section is to show how to run the compass checkers on a whole project (like e.g ROSE) 
using a one button click in the Compass GUI and present an easy interface for exploring the violations
as found during the build. This interface is currently limited to sequentially building the
project and it requires ROSE to be configured with sqlite3. The Compass GUI does not try to replace the 
build system; it simply captures how the build system compiles the source files for a specific version of the 
project. Although there is no guarantee that this will work when the source code changes it is reasonable to 
expect the capturing of the build system should be the same as the code evolves as long as no changes are done 
to the build system.

These instructions can apply to Autotool projects as well as projects build in other build 
systems, but the shorthands used here for discerning the build system are specific for Autotools.


\subsection{Capturing a Build System State}

\fixme{Document: that this requires --with-sqlite3}

The first step of running the Compass GUI on an autotools project is to figure out how the build system
works. Capturing the build system state is done with the 'buildInterpreter' tool provided in the Compass
distribution in the 'projects/compass/tools/compass/buildInterpreter' directory. 

In order to facilitate capturing the build system once and moving the source files around the '\$ROSE\_TEST\_REGRESSION\_ROOT' environment variable must be defined to the string that should be replaced. For instance if a
projects is build within /home/user/project and it is desired to move the files inside that directoy to a
different directory define it to be
\begin{verbatim}
export ROSE_TEST_REGRESSION_ROOT=/home/user/project
\end{verbatim}

\fixme{Document: Need to be in the buildInterpreter directory.}

The buildInterpreter tool works as a replacement for the C/C++ compiler during compilation like e.g:
\begin{verbatim}
buildInterpreter -o ex1 ex1.C
\end{verbatim}
The output of the run is a database representing how to compile ex1.C. The name of the output database is specified
by the 'dbname' field in the rqmc file found in the ROSE build directory under 'projects/compass/tools/compass/buildInterpreter/rqmc' and defaults to 'test.db'.

To capture the state of a whole build system use 'buildInterpreter' as a replacement for the C/C++ compiler during
the build. E.g for gnu make:
\begin{verbatim}
make CC=buildInterpreter CXX=buildInterpreter
\end{verbatim}

\subsection{Build A Project Using the Discerned Build }

To build a project using Compass specify the output database from the capturing of the build state with the
'--outputDb' paramater to the Compass GUI. The environment variable must be defined like in \ref{runningCompassGUISingleFile}, e.g:
\begin{verbatim}
cd /directory/with/build/project/sources
env COMPASS_PARAMETERS=/location/of/compass_pramateres   /compass/gui/build/compassMainGui --outputDb test.db
\end{verbatim}
In the GUI click on regenerate to build the project. The violations found during the build is put into the database
for subsequent lookup. After regenerating select the checkers that you are interested and and click 'refresh' to display the corresponding violations.




