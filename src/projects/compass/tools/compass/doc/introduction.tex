\chapter{Introduction}

\quote{A software analysis story about four people named: Everybody, Somebody, Anybody, and Nobody.
There was an important job to be done and Everybody was sure that Somebody would do it.
Anybody could have done it, but Nobody did it.  Somebody got angry about that because
it was Everybody's job.  Everybody thought that Anybody could do it, but Nobody
realized that Everybody wouldn't do it. It ended up that Everybody blamed Somebody
when Nobody did what Anybody could have done.}
\footnote{Story not really specific to softare analysis found at
 http://www.corsinet.com/braincandy/hlife.html}


\section{Overview}

\label{introduction::overview}

   Compass is a tool for the checking of source code.  It is
based on the ROSE compiler infrastructure and demonstrates to
use of ROSE to build lots of simple pattern detectors for analysis
of C, C++, and Fortran source code.

   The purpose of this work is several fold:
\begin{itemize}
   \item Provide a concrete tool to support interactions with lab customers.
   \item Provide a home for the security analysis specific detectors being built within
         external research projects.
   \item Provide an external tool for general analysis of software.
   \item Provide a tool to support improvements to the ROSE source code base.
   \item Define an infrastructure for an evolving and easily tailored program analysis tool.
   \item Provide a simple motivation for expanded use of ROSE by external users.
         Development, testing, and evaluation of ROSE infrastructure is best facilitated 
         through its expanded use by others and this provides a specific and attractive
         tool that can provide feedback to users about their own code projects.  Even
         though optimization research is our focus, this gets our supporting
         infrastructure for optimization research out and into use by others in the form
         of an extensible tool.
\end{itemize}

   Note that as the collection of detectors grows we will periodically reorganize the 
collection.  At some point soon we will build a hierarchy to organize the evolving
collection.

%\subsection{A basis for other source analysis tools}
\paragraph{A basis for other source analysis tools}
   Input and output to ROSE is organized so that any number of sources could be used.
So although we provide a compiler interface (for simplicity), we will also provide a 
GUI interface as an alternative interface to demonstrate that the detectors are orthogonal
to there use in alternative tools.  Alternative tool interfaces should be possible 
and will further demonstrate the independence of the input and output mechanisms to
the designs and implementation of the core detectors.

\paragraph{Add Your Own Detector}

    Detectors written in Compass make direct use of ROSE and are 
designed to be copied and extended by users to develop their own 
detectors. We welcome the contribution of these detectors back to 
the ROSE team for inclusion into future releases of Compass;
full credit for all work will be provide to all authors.
Compass is an open source project using ROSE, an open source
compiler infrastructure.

    Each of the detectors are examples of how to
add your own detector to {\bf Compass}.  If you
build a detector that you would like to have be 
distributed with {\bf Compass}, please send it to
us and we will add your as an external contributor.

  Guidelines for contributions:
\begin{itemize}
   \item Use any Compass detector and an example.
   \item provide the documentation about your detector.
   \item Use any features in ROSE to support your detector; AST, Control Flow graph,
    System dependence Graph, Call Graph, Class Hierarchy Graph, etc.
   \item Your detector should have {\bf NO} side-effects on the AST.
\end{itemize}





