\clearpage
\section{Appendix}
This section gives quick instructions for installing some needed software packages to establish the end-to-end autotuning system. 
Please consult individual software releases for detailed installation guidance. 

\subsection{Patching Linux Kernels with perfctr}
This is required before you can install PAPI on Linux/x86 and Linux/x86\_64 platforms. 
Take PAPI 3.6.2 as an example, the steps to patch your kernel are:
\begin{itemize}
\item Find the latest perfctr patch which matches your Linux distribution from \textit{papi-3.6.2/src/perfctr-2.6.x/patches}. 
For Red Hat Enterprise Linux 5 (or CentOS 5), the latest kernel patch is \textit{patch-kernel-2.6.18-92.el5-redhat}.
\item Get the Linux kernel source rpm package which matches the perfctr kernel patch found in
the previous step. 
You can find kernel source rpm packages from one of the many mirror sites. For example, 
\textit{wget http://altruistic.lbl.gov/mirrors/centos/5.2/updates/SRPMS/kernel-2.6.18-92.1.22.el5.src.rpm}
\item Install the kernel source rpm package. With a root privilege, simply type: 
\begin{verbatim}
rpm -ivh kernel*.src.rpm
\end{verbatim}
The command will generate a set of patch files under
\textit{/usr/src/redhat/SOURCES}.
\item Generate the kernel source tree from the patch files. 
This step may require the \textit{rpm-build} and \textit{redhat-rpm-config} packages to be installed first. 
\begin{verbatim}
yum install rpm-build redhat-rpm-config      # with the root privilege 
cd /usr/src/redhat/SPECS
rpmbuild -bp --target=i686 kernel-2.6.spec   # for x86 platforms    
rpmbuild -bp --target=x86_64 kernel-2.6.spec #for x86_64 platforms
\end{verbatim}

\item Copy the kernel source files and create a soft link. Type 
\begin{verbatim}
cp -a /usr/src/redhat/BUILD/kernel-2.6.18/linux-2.6.18.i686 /usr/src
ln -s /usr/src/linux-2.6.18.i686 /usr/src/linux
\end{verbatim}

\item Now you can patch the kernel source files to support perfctr. Type 
\begin{verbatim}
cd /usr/src/linux
/path/to/papi-3.6.2/src/perfctr-2.6.x/update-kernel \
     --patch=2.6.18-92.el5-redhat
\end{verbatim}

\item Configure your kernel to support hardware counters.
\begin{verbatim}
cd /usr/src/linux 
make clean
make mrproper 
yum install ncurses-devel 
make menuconfig
\end{verbatim}
Enable all items for \textit{performance-monitoring counters support} under the menu item of \textit{processor type and features}.
\item Build and install your patched kernel. 
\begin{verbatim}
make -j4 && make modules -j4 && make modules_install && make install
\end{verbatim}
\item Configure perfctr as a device that is automatically loaded each time you boot up your machine.
\begin{verbatim}
cd /home/liao6/download/papi-3.6.2/src/perfctr-2.6.x
cp etc/perfctr.rules /etc/udev/rules.d/99-perfctr.rules
cp etc/perfctr.rc /etc/rc.d/init.d/perfctr
chmod 755 /etc/rc.d/init.d/perfctr
/sbin/chkconfig --add perfctr
\end{verbatim}
\end{itemize}
With the kernel patched, it is straightforward to install PAPI. 
\begin{verbatim}
cd /home/liao6/download/papi-3.6.2/src
./configure
make
make test
make install # with a root privilege
\end{verbatim}
%----------------------------------------------
\subsection{Installing BLCR}
BLCR (the Berkeley Lab Checkpoint/Restart library)'s installation guide can be found at
\url{http://upc-bugs.lbl.gov/blcr/doc/html/BLCR_Admin_Guide.html}.
We complement the guide with some Linux-specific information here.
It is recommended to use a separate build tree to compile the library.
\begin{verbatim}
mkdir buildblcr
cd buildblcr
# explicitly provide the Linux kernel source path 
../blcr-0.8.2/configure --with-linux=/usr/src/linux-2.6.18.i686
make
# using a root account
make install
make insmod check

# Doublecheck kernel module installation 
# You should find two module files: blcr_imports.ko blcr.ko 
ls /usr/local/lib/blcr/2.6.18-prep/ 
\end{verbatim}
To configure your system to load BLCR kernel modules at bootup:
\begin{verbatim}
# copy the sample service script to the right place
cp blcr-0.8.2/etc/blcr.rc /etc/init.d/.
# change the module path inside of it 
vi /etc/init.d/blcr.rc
  #module_dir=@MODULE_DIR@
  module_dir=/usr/local/lib/blcr/2.6.18-prep/
#run the blcr service each time you boot up your machine
chkconfig --level 2345 blcr.rc on

# manually start the service 
# error messages like "FATAL: Module blcr not found." can be ignored. 
/etc/init.d/blcr.rc restart
Unloading BLCR: FATAL: Module blcr not found.
FATAL: Module blcr_imports not found.
                                                           [  OK  ]
Loading BLCR: FATAL: Module blcr_imports not found.
FATAL: Module blcr not found.
                                                           [  OK  ]

\end{verbatim}


