
\chapter{ Glossary }

%  Purpose:
% \begin{itemize}
%    \item A. Define terminology
% \end{itemize}
% \begin{center}
% *********************  \newline
% \end{center}
% \vspace{0.25in}

\fixme{Define the following terms: IR node, Inherited Attribute, Synthesized Attribute,
    Accumulator Attribute, AST Traversal }

    We define terms used in the ROSE manual which might otherwise be unclear.

\begin{itemize}
   \item {\bf AST} Abstract Syntax Tree. A very basic understanding of an AST is the
                   entry level into ROSE.

   \item {\bf Attribute} User defined information (objects) associated with IR nodes.
    Forms of attributes include: accumulator, inherited, persistent, and synthesized.
    Both inherited and synthesized attributes are managed automatically on the stack within a
    traversal.  Accumulator attributes are typically something semantically equivalent to a
    global variable (often a static data member of a class).  Persistent attributes 
    are explicitly added to the AST and are managed directly by the user.  As a result,
    they can persist across multiple traversals of the AST.  Persistent attributes are
    also saved in the binary file I/O, but only if the user provides the attribute
    specific {\tt pack()} and {\tt unpack()} virtual member functions.  See the ROSE
    User Manual for more information, and the ROSE Tutorial for examples.

   \item {\bf CFG} As used in ROSE, this is the Control Flow Graph, not Context Free
    Grammar or anything else.

   \item {\bf EDG} Edison Design Group (the commercial company that produces the C and C++
    front-end that is used in ROSE).

   \item {\bf IR} Intermediate Representation (IR).  The IR is the set of classes defined
    within SAGE III that allow an AST to be built to define any application in C, C++,
    and Fortran application.

   \item {\bf Query} (as in AST Query) Operations on the AST that return answers to
    questions posed about the content or context in the AST.

   \item {\bf ROSE} A project that covers both research in optimization and a specific
    infrastructure for handling large scale C, C++, and Fortran applications.

   \item {\bf Rosetta} A tool (written by the ROSE team) used within ROSE to automate the
    generation of the SAGE III IR.

   \item {\bf SAGE++ and SAGE II} An older object-oriented IR upon which the API of 
   SAGE III IR is based.

   \item {\bf Semantic Information} What abstractions mean (short answer).
    (This might be better as a description of what kind of semantic information ROSE could
    take advantage, not a definition.)

   \item {\bf Telescoping Languages} A research area that defines a process to generate
    domain-specific languages from a general purpose languages.

   \item {\bf Transformation} The process of automating the editing (either
    reconfiguration, addition, or deletion; or some combination) of input
    application parts to build a new application.  In the context of ROSE, all
    transformations are source-to-source.

   \item {\bf Translator} An executable program (in our context built using ROSE) that performs
    source-to-source translation on an existing input application source to generate 
    a second (generated) source code file.  The second (generated) source code is 
    then typically provided as input to a vendor provided compiler (which generates
    object code or an executable program).

   \item {\bf Traversal} The process of operating on the AST in some order (usually
    pre-order, post-order, out of order [randomly], depending on the traversal that is
    used). The ROSE user builds a traversal from base classes that do the
    traversal and execute a function, or a number of functions, provided by the user.

\end{itemize}






