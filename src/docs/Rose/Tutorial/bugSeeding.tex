\chapter{Bug Seeding}

  Bug seeding is a technique used to construct example codes from 
existing codes which can be used to evaluate tools for the finding
bugs in randomly selected existing applications.  The idea is to
seed an existing application with a known number of known bugs
and evaluate the bug finding or security tool based on the
percentage of the number of bugs found by the tool.  If the
bug finding tool can identify all known bugs then there can
be some confidence that the tool detects all bugs of that type
used to seed the application.

   This example tutorial code is a demonstration of a more
complete technique being developed in collaboration with NIST 
to evaluate tools for finding security flaws in applications.
It will in the future be a basis for testing of tools built using
ROSE, specifically Compass, but the techniques are not in any
way specific to ROSE or Compass.

\section{Input For Examples Showing Bug Seeding}

   Figure~\ref{Tutorial:exampleInputCode_bugSeeding}
shows the example input used for demonstration of bug seeding as 
a transformation.

\begin{figure}[!h]
{\indent
{\mySmallFontSize

% Do this when processing latex to generate non-html (not using latex2html)
\begin{latexonly}
   \lstinputlisting{\TutorialExampleDirectory/inputCode_seedBugsExample_arrayIndexing.C}
\end{latexonly}

% Do this when processing latex to build html (using latex2html)
\begin{htmlonly}
   \verbatiminput{\TutorialExampleDirectory/inputCode_seedBugsExample_arrayIndexing.C}
\end{htmlonly}

% end of scope in font size
}
% End of scope in indentation
}
\caption{Example source code used as input to program in
         codes used in this chapter.}
\label{Tutorial:exampleInputCode_seedBugs}
\end{figure}


\section{Generating the code representing the seeded bug}

    Figure~\ref{Tutorial:example_seedBugs}
shows a code that traverses each IR node and for and
modifies array reference index expressions to be out of bounds.
The input code is shown in figure \ref{Tutorial:exampleInputCode_seedBugs},
the output of this code is shown in 
figure~\ref{Tutorial:exampleOutput_seedBugs}.


\begin{figure}[!h]
{\indent
{
%\mySmallFontSize
\mySmallestFontSize

% Do this when processing latex to generate non-html (not using latex2html)
\begin{latexonly}
   \lstinputlisting{\TutorialExampleDirectory/seedBugsExample_arrayIndexing.C}
\end{latexonly}

% Do this when processing latex to build html (using latex2html)
\begin{htmlonly}
   \verbatiminput{\TutorialExampleDirectory/seedBugsExample_arrayIndexing.C}
\end{htmlonly}

% end of scope in font size
}
% End of scope in indentation
}
\caption{Example source code showing how to seed bugs. }
\label{Tutorial:example_seedBugs}
\end{figure}


\begin{figure}[!h]
{\indent
{\mySmallFontSize

% Do this when processing latex to generate non-html (not using latex2html)
\begin{latexonly}
   \lstinputlisting{\TutorialExampleBuildDirectory/rose_inputCode_seedBugsExample_arrayIndexing.C}
\end{latexonly}

% Do this when processing latex to build html (using latex2html)
\begin{htmlonly}
   \verbatiminput{\TutorialExampleBuildDirectory/rose_inputCode_seedBugsExample_arrayIndexing.C}
\end{htmlonly}

% end of scope in font size
}
% End of scope in indentation
}
\caption{Output of input code using seedBugsExample\_arrayIndexing.C}
\label{Tutorial:exampleOutput_seedBugs}
\end{figure}

