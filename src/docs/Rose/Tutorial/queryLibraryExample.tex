\chapter{AST Query}

   This chapter presents a mechanism for simple queries on the AST.
Such queries are typically a single line of code, instead of the 
class that must be declared and defined when using the traversal mechanism.
While the traversal mechanism is more sophisticated and more powerful, the AST Query
mechanism is particularly simple to use.

\section{Simple Queries on the AST}

   This section demonstrates a simple query on the AST.

\begin{figure}[!h]
{\indent
{\mySmallFontSize


% Do this when processing latex to generate non-html (not using latex2html)
\begin{latexonly}
   \lstinputlisting{\TutorialExampleDirectory/queryLibraryExample.C}
\end{latexonly}

% Do this when processing latex to build html (using latex2html)
\begin{htmlonly}
   \verbatiminput{\TutorialExampleDirectory/queryLibraryExample.C}
\end{htmlonly}

% end of scope in font size
}
% End of scope in indentation
}
\caption{Example source code for translator to read an input program and 
         generate a list of functions in the AST (queryLibraryExample.C).}
\label{Tutorial:exampleQueryLibrary}
\end{figure}

\begin{figure}[!h]
{\indent
{\mySmallFontSize


% Do this when processing latex to generate non-html (not using latex2html)
\begin{latexonly}
   \lstinputlisting{\TutorialExampleDirectory/inputCode_QueryLibrary.C}
\end{latexonly}

% Do this when processing latex to build html (using latex2html)
\begin{htmlonly}
   \verbatiminput{\TutorialExampleDirectory/inputCode_QueryLibrary.C}
\end{htmlonly}

% end of scope in font size
}
% End of scope in indentation
}
\caption{Example source code used as input to program in
    figure~\ref{Tutorial:exampleQueryLibrary} (queryLibraryExample.C).}
\label{Tutorial:exampleInputCode_QueryLibrary}
\end{figure}

% (09/10/2019) Pei-Hung remove the following figure because it cannot be generated into PDF
\commentout{
\begin{figure}[!h]
{\indent
{\mySmallFontSize


% Do this when processing latex to generate non-html (not using latex2html)
\begin{latexonly}
   \lstinputlisting{\TutorialExampleBuildDirectory/queryLibrary.out}
\end{latexonly}

% Do this when processing latex to build html (using latex2html)
\begin{htmlonly}
   \verbatiminput{\TutorialExampleBuildDirectory/queryLibrary.out}
\end{htmlonly}

% end of scope in font size
}
% End of scope in indentation
}
\caption{Output of input file to the AST query processor (queryLibraryExample.C).}
\label{Tutorial:exampleOutput_QueryLibrary}
\end{figure}
}

The program in figure~\ref{Tutorial:exampleQueryLibrary} calls 
an internal ROSE Query Library.  Queries of the AST using the query library are
particularly simple and often are useful as nested queries within more complex analysis.
More information of the ROSE AST Query Library is available within ROSE User Manual.
% chapter~\ref{QueryLibrary:QueryLibrary}.

% (09/10/2019) Pei-Hung remove the following because example output cannot be generated
\commentout{
   Using the input program in figure~\ref{Tutorial:exampleInputCode_QueryLibrary}
the translator processes the code and generates the output in 
figure~\ref{Tutorial:exampleOutput_QueryLibrary}.
}
\fixme{Put an example of composition of AST queries into the example input code.}

\section{Nested Query}

   This section demonstrates a nested AST query, showing how to use 
composition in the construction of more elaborate queries from simple ones.

\begin{figure}[!h]
{\indent
{\mySmallFontSize


% Do this when processing latex to generate non-html (not using latex2html)
\begin{latexonly}
   \lstinputlisting{\TutorialExampleDirectory/nestedQueryExample.C}
\end{latexonly}

% Do this when processing latex to build html (using latex2html)
\begin{htmlonly}
   \verbatiminput{\TutorialExampleDirectory/nestedQueryExample.C}
\end{htmlonly}

% end of scope in font size
}
% End of scope in indentation
}
\caption{Example source code for translator to read an input program and 
         generate a list of access functions in the AST (nestedQueryExample.C).}
\label{Tutorial:exampleNestedQuery}
\end{figure}

\begin{figure}[!h]
{\indent
{\mySmallFontSize

% Do this when processing latex to generate non-html (not using latex2html)
\begin{latexonly}
%  \lstinputlisting{\TutorialExampleDirectory/inputCode_NestedQuery.C}
   \lstinputlisting{\TutorialExampleDirectory/inputCode_QueryLibrary.C}
\end{latexonly}

% Do this when processing latex to build html (using latex2html)
\begin{htmlonly}
%  \verbatiminput{\TutorialExampleDirectory/inputCode_NestedQuery.C }
   \verbatiminput{\TutorialExampleDirectory/inputCode_QueryLibrary.C }
\end{htmlonly}

% end of scope in font size
}
% End of scope in indentation
}
\caption{Example source code used as input to program in
    figure~\ref{Tutorial:exampleNestedQuery} (nestedQueryExample.C).}
\label{Tutorial:exampleInputCode_NestedQuery}
\end{figure}

\begin{figure}[!h]
{\indent
{\mySmallFontSize


% Do this when processing latex to generate non-html (not using latex2html)
\begin{latexonly}
   \lstinputlisting{\TutorialExampleBuildDirectory/nestedQuery.out}
\end{latexonly}

% Do this when processing latex to build html (using latex2html)
\begin{htmlonly}
   \verbatiminput{\TutorialExampleBuildDirectory/nestedQuery.out}
\end{htmlonly}

% end of scope in font size
}
% End of scope in indentation
}
\caption{Output of input file to the AST query processor (nestedQueryExample.C).}
\label{Tutorial:exampleOutput_NestedQuery}
\end{figure}

The number of traversals of the AST can be reduced by using nested queries. 
Nested queries permits queries on the result from a NodeQuery. Another advantage is 
that nested (combined) queries can be formed to query for information without writing new 
query, the nested query is a new query. 

The program in figure~\ref{Tutorial:exampleNestedQuery} calls 
an internal ROSE Query Library. Two different queries are performed to find all
access functions within the AST. The first query is nested, the 
returned list from a query is used in a traversal, and the second query queries the AST
for the same nodes.

Using the input program in figure~\ref{Tutorial:exampleInputCode_NestedQuery}
the translator processes the code and generates the output in 
figure~\ref{Tutorial:exampleOutput_NestedQuery}.

