\documentclass[times, 10pt]{book}
 
%\usepackage{akit/latex8}
\usepackage{times}
\usepackage{epsfig}
\usepackage{subfigure}
 
\long\def\com#1{}
 
 
\begin{document}
 
\date{}
 
\chapter{Program Analysis}

\section{Database Support for Global Analysis}
\label{RoseExamples:RoseTutorial:Database}
 
 
\subsection{Database Setup}
\label{sec:DatabaseRequirements}

ROSE uses a MySQL database to store global information over
multiple runs of a preprocessor.  If a MySQL database and it's C++ 
interface MySQL++ are installed,
you can skip the installation instructions. 
However first ensure that connection parameters such as hostname and 
login are correctly specified.

These two packages are not distributed with ROSE.  To install them
simply download the current production release of MySQL and
the latest version of the MySQL++ API from the download section
of http://www.mysql.org, then configure and install first MySQL then
MySQL++. Under Unix, both can easily be installed locally into a home directory
without requiring root access. 

The following subsections describe MySQL installation and configuration.
Neither are intended to replace the MySQL manual, but should provide
sufficient information to configure the ROSE database support.

\subsubsection{MySQL installation}

This section describes installation of MySQL and the MySQL++ library.
If these installations are already present on your section, you
make skip this section.

Download the current release of MySQL and MySQL++ from www.mysql.org.  
Please read the {\tt README} in the MySQL++ distribution, which states
that building with a GNU compiler requires a version greater than 2.95.
Further there are 3 patches (for gcc 3.0, 3.2, and 3.2.2) available.

Though not a requirement, we assume that MySQL and the MySQL++ library
installation files (including patches for GCC 3.0, 3.2, and 3.2.2) have been
copied to \textless{}CompileTree\textgreater{}/src/3rdPartyLibraries.  

These installation instructions assume the use of GCC 3.2.2.

To install MySQL execute the following commands:

\begin{verbatim}
cd <CompileTree>/src/3rdPartyLibraries
tar xvfz mysql-4.0.20.tar.gz
cd mysql-4.0.20

# create a MySQL install dir
mkdir <CompileTree>/src/3rdPartyLibraries/MySQL_Install
configure --prefix=<CompileTree>/src/3rdPartyLibraries/MySQL_Install

# (parallel) build
make -j2
# install
make install
\end{verbatim}

To install the MySQL++ library execute the following commands:

\begin{verbatim}
cd <CompileTree>/src/3rdPartyLibraries
tar xvfz mysql++-1.7.9.tar.gz
gunzip mysql++-gcc-3.0.patch.gz
gunzip mysql++-gcc-3.2.patch.gz
gunzip mysql++-gcc-3.2.2.patch.gz
cd mysql++-1.7.9

# apply the patch as appropriate for your compiler
# each patch is dependent on the patch for the previous version 
# of the compiler
patch -p 1 < ../mysql++-gcc-3.0.patch
patch -p 1 < ../mysql++-gcc-3.2.patch
patch -p 1 < ../mysql++-gcc-3.2.2.patch

# rerun automake to rebuild Makefile.in files after running patch
# need to use automake version 1.4 
# (download from ftp.gnu.org/gnu/automake)
automake -a
autoconf

# configure MySQL++
./configure --prefix=<CompileTree>/src/3rdPartyLibraries/MySQL_Install \
            --with-mysql=<CompileTree>/src/3rdPartyLibraries/MySQL_Install \
            --with-mysql-lib=<CompileTree>/src/3rdPartyLibraries/MySQL_Install

# build
make

# install
make install
\end{verbatim}

\subsubsection{MySQL configuration}

This section describes configuration and setup of a MySQL database daemon,
required for the use of ROSE's database facilities.

ROSE must be configured with the location of the MySQL libraries, as well
as a MySQL username, server, password, and database name.  ROSE can be 
re-configured with these options after an initial reconfiguration.  In
that case, it should be rebuilt.  Rebuilding will only build the
database and anything dependent upon it.

For example:

\begin{verbatim}
<SrcTree>/configure --with-MySQL=<CompileTree>/src/3rdPartyLibraries/MySQL_Install \
                    --with-MySQL_username=joe \
                    --with-MySQL_server=`hostname` \
                    --with-MySQL_password=rosepwd \
                    --with-MySQL_database_name=databasejoe
\end{verbatim}

The previous configuration assumes that the developer will start a database
on the local host.  If one is used elsewhere, the corresponding hostname
should be specified above.

Depending on the location of the MySQL installation, a ROSE user may or
may not have write access to the default 'datadir' in which {\tt mysqld}
attempts to store data.  While these permissions problems will likely not
be an issue if the user has followed the above installation instructions,
in the following we assume that the user will specify some alternative
directory (e.g., /tmp/mysqld-datadir) to store databases.

Upon initial installation, the user must install certain administrative
databases.  This is accomplished by executing the {\tt mysql\_install\_db}
script.  After performing this initial setup, the user may
execute the database daemon {\tt mysqld\_safe}.  

By default the MySQL {\tt root} user has privilege access to the database.
A user can gain privileged access by simply connecting as the {\tt root}
user, without specifying a password.  It should be noted that MySQL
user accounts are not equivalent to Unix user accounts.  The former are
specified via the {\tt -u} or {\tt --user=} flags to MySQL commands,
though they default to the Unix user that executes them.  Immediately
after installation and daemon startup, the user should set the MySQL
{\tt root} password via the {\tt mysqladmin} command.  

As configured above with the {\tt --with-MySQL*} flags, ROSE applications
will attempt to access the database as user {\tt joe}, supplying
password {\tt rosepwd}.  By default {\tt joe} will not have the required
permissions to access the database.  The MySQL {\tt root} user can
perform access control via the {\tt GRANT} and {\tt REVOKE} commands.
Please see the MySQL manual section on {\tt GRANT} and {\tt REVOKE} syntax
for more details.

The following is intended as a guideline of a minimal configuration.  Please
see the chapter on 'Installing MySQL' in the MySQL database for more
detailed documentation.

\begin{verbatim}

# create a datadir
mkdir /tmp/mysqld-datadir

# instantiate administrative databases
$MYSQL_INSTALL_DIR/bin/mysql_install_db --datadir=/tmp/mysqld-datadir/

# start the daemon
$MYSQL_INSTALL_DIR/bin/mysqld_safe --datadir=/tmp/mysqld-datadir/ &

# set the root password
$MYSQL_INSTALL_DIR/bin/mysqladmin --user=root password rootpwd

# connect to the database as root to perform access control
$MYSQL_INSTALL_DIR/bin/mysql --user=root --password=rootpwd mysql

# ... at the mysql prompt do:

# remove default MySQL configuration so that username is required
mysql> DELETE FROM user WHERE user='';

# adds new user 'joe' with password 'rosepwd' with access to all tables
mysql> GRANT ALL ON *.* TO joe IDENTIFIED BY 'rosepwd';

# forces MySQL to reload privileged information (user table in MySQL)
mysql> FLUSH PRIVILEGES;

\end{verbatim}


\end{document}
